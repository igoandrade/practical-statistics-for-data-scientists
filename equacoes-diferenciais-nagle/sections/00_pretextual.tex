% ---
% Capa
% ---
\imprimircapa
% ---

% ---
% Folha de rosto
% (o * indica que haverá a ficha bibliográfica)
% ---
%\imprimirfolhaderosto
% ---

% ---
% Inserir a ficha bibliografica
% ---

% Isto é um exemplo de Ficha Catalográfica, ou ``Dados internacionais de
% catalogação-na-publicação''. Você pode utilizar este modelo como referência. 
% Porém, provavelmente a biblioteca da sua universidade lhe fornecerá um PDF
% com a ficha catalográfica definitiva após a defesa do trabalho. Quando estiver
% com o documento, salve-o como PDF no diretório do seu projeto e substitua todo
% o conteúdo de implementação deste arquivo pelo comando abaixo:
%
% \begin{fichacatalografica}
%     \includepdf{fig_ficha_catalografica.pdf}
% \end{fichacatalografica}

\begin{comment}
\begin{fichacatalografica}
    \begin{center}
        \chaptitlefont PLANO DE TRABALHO - ESTÁGIO II
    \end{center}
    \begin{table}[H]
        \centering
        \begin{tabular}{p{5cm} p{10cm}}
            \hline \\
            \textbf{Empresa} &  \ORG \\
            \textbf{Endereço} &  Quadra 104, Rua SE 11 35 Conj 03, Lt 33, Sala 04 \\
            \textbf{Setor/Departamento} &  Contábil \\
            \textbf{Horário do Estágio} &  08:00 h as 14:00 h\\
            \textbf{Supervisor} &  Geone Barbosa de Assis \\ \\
            \hline \\
            \textbf{Estagiário} &  Igo da Costa Andrade \\
            \textbf{Nº de Matrícula} &  2017110793 \\
            \textbf{Curso} &  Ciências Contábeis \\
            \textbf{Vigência do Estágio} &  01/04/2022 a 02/05/2022 \\
            \textbf{Carga Horária} &  120 horas \\ \\ \hline 
        \end{tabular}
    \end{table}
    \textbf{Atividades Programadas para o Estágio:}
    \begin{itemize}
        \item Organização e classificação contábil de documentos das empresas clientes do escritório
        \item Geração de lançamentos contábeis
        \item Realizar a conciliação de contas bancárias
        \item Arquivamento de documentos
    \end{itemize}
\end{fichacatalografica}
\newpage
\end{comment}
% ---


% ---
% Inserir folha de aprovação
% ---

% Isto é um exemplo de Folha de aprovação, elemento obrigatório da NBR
% 14724/2011 (seção 4.2.1.3). Você pode utilizar este modelo até a aprovação
% do trabalho. Após isso, substitua todo o conteúdo deste arquivo por uma
% imagem da página assinada pela banca com o comando abaixo:
%
% \begin{folhadeaprovacao}
% \includepdf{folhadeaprovacao_final.pdf}
% \end{folhadeaprovacao}
%
\begin{comment}
\begin{folhadeaprovacao}

  \begin{center}
    {\ABNTEXchapterfont\large\imprimirautor}

    \vspace*{\fill}\vspace*{\fill}
    \begin{center}
      \ABNTEXchapterfont\bfseries\large\imprimirtitulo
    \end{center}
    \vspace*{\fill}
    
    \hspace{.45\textwidth}
    \begin{minipage}{.5\textwidth}
        \imprimirpreambulo
    \end{minipage}%
    \vspace*{\fill}
   \end{center}

    \noindent Data de Aprovação:
    \\ \\
    \noindent Banca Examinadora:

   \assinatura{\textbf{\imprimirorientador} \\ Orientador} 
   \assinatura{\textbf{Professor} \\ Examinador 1}
   \assinatura{\textbf{Professor} \\ Examinador 2}
   %\assinatura{\textbf{Professor} \\ Convidado 3}
   %\assinatura{\textbf{Professor} \\ Convidado 4}
  
\end{folhadeaprovacao}
\end{comment}

% ---

% ---
% Dedicatória
% ---
\begin{comment}
\begin{dedicatoria}
   \vspace*{\fill}
   \begin{flushright}
       \noindent
       \textit{ Este trabalho é dedicado às crianças adultas que,\\
       quando pequenas, sonharam em se tornar cientistas.}
   \end{flushright}
\end{dedicatoria}
\end{comment}

% ---

% ---
% Agradecimentos
% ---

\begin{comment}
\begin{agradecimentos}
Os agradecimentos principais são direcionados à Gerald Weber, Miguel Frasson,
Leslie H. Watter, Bruno Parente Lima, Flávio de Vasconcellos Corrêa, Otavio Real
Salvador, Renato Machnievscz\footnote{Os nomes dos integrantes do primeiro
projeto abn\TeX\ foram extraídos de
\url{http://codigolivre.org.br/projects/abntex/}} e todos aqueles que
contribuíram para que a produção de trabalhos acadêmicos conforme
as normas ABNT com \LaTeX\ fosse possível.

Agradecimentos especiais são direcionados ao Centro de Pesquisa em Arquitetura
da Informação\footnote{\url{http://www.cpai.unb.br/}} da Universidade de
Brasília (CPAI), ao grupo de usuários
\emph{latex-br}\footnote{\url{http://groups.google.com/group/latex-br}} e aos
novos voluntários do grupo
\emph{\abnTeX}\footnote{\url{http://groups.google.com/group/abntex2} e
\url{http://www.abntex.net.br/}}~que contribuíram e que ainda
contribuirão para a evolução do \abnTeX.

\end{agradecimentos}
\end{comment}

% ---

% ---
% Epígrafe
% ---
\begin{comment}
\begin{epigrafe}
    \vspace*{\fill}
	\begin{flushright}
		\textit{``Não vos amoldeis às estruturas deste mundo, \\
		mas transformai-vos pela renovação da mente, \\
		a fim de distinguir qual é a vontade de Deus: \\
		o que é bom, o que Lhe é agradável, o que é perfeito.\\
		(Bíblia Sagrada, Romanos 12, 2)}
	\end{flushright}
\end{epigrafe}
\end{comment}

% ---

% ---
% RESUMOS
% ---
% resumo em português
%\setlength{\absparsep}{10pt} % ajusta o espaçamento dos parágrafos do resumo
\begin{comment}
\begin{resumo}
    \begin{SingleSpace}
        Relatório das atividades desenvolvidas dentro do escritório de contabilidade conforme proposto na disciplina de Estágio II, que teve por objetivo descrever a aplicação prática das rotinas relacionadas ao Departamento Contábil dentro de uma organização do ramo contábil. Este relatório visa resumir as atividades e procedimentos da atividade contábil e demonstrar os resultados das atividades práticas das organizações. As atividades foram apresentadas pelo contador responsável pela empresa ORG  Contabilidade e supervisionado pelo professor Msc. \imprimirorientador, em atendimento a Lei de Estágio nº 11.788/2008, visando a integração entre a atividade teórica e prática das competências próprias da atividade profissional ligadas ao Departamento Pessoal da organização de acordo com roteiro estabelecido através do Plano de Atividades do Estágio.\\
    
        \textbf{Palavras-chave}: Estágio Supervisionado. Ciências Contábeis. Departamento Contábil.
    \end{SingleSpace}
\end{resumo}
\end{comment}
% ---

% ---
% inserir lista de ilustrações
% ---
\pdfbookmark[0]{\listfigurename}{lof}
\listoffigures*
\cleardoublepage
% ---

% ---
% inserir lista de quadros
% ---
%\pdfbookmark[0]{\listofquadrosname}{loq}
%\listofquadros*
%\cleardoublepage
% ---

% ---
% inserir lista de tabelas
% ---

\pdfbookmark[0]{\listtablename}{lot}
\listoftables*
\cleardoublepage

% ---

% ---
% inserir lista de abreviaturas e siglas
% ---
\begin{comment}
\begin{siglas}
    \item[CNPJ] Cadastro Nacional de Pessoas Jurídicas
    \item[DCN] Diretrizes Curriculares Nacionais
    \item[RFB] Receita Federal do Brasil
    \item[UFT] Universidade Federal do Tocantins
\end{siglas}
\end{comment}
% ---

% ---
% inserir lista de símbolos
% ---
\begin{comment}
\begin{simbolos}
  \item[$ \Gamma $] Letra grega Gama
  \item[$ \Lambda $] Lambda
  \item[$ \zeta $] Letra grega minúscula zeta
  \item[$ \in $] Pertence
\end{simbolos}
\end{comment}

% ---


% ---
% inserir o sumario
% ---
\pdfbookmark[0]{\contentsname}{toc}
\tableofcontents*
\cleardoublepage
% ---


